\documentclass[a4paper,10pt]{article}
%\usepackage[utf8x]{inputenc}
\usepackage[utf8]{inputenc}
\usepackage{graphicx}

%opening
\title{Automatic detection of bad habits of programming in Scratch}
\author{Jesús Moreno, Gregorio Robles}

\begin{document}

\maketitle

\begin{abstract}
This paper describes a method for automatic detection of two very bad programming habits present in students learning to program with Scratch, such as the repetition of code and object naming, and raises ideas to try to avoid such situations.

\end{abstract}

\section{Introduction}
Scratch \cite{resnick2009scratch}  is a programming environment that includes a visual programming language designed for children from 6 years old, a development environment and a website where the community can host their projects; run, study and reuse other programs; and share ideas or suggestions with other programmers. Although there are several similar visual programming languages, Scratch is undoubtedly the most successful and the one that is being used more regularly worldwide, with more than three million registered users on its website and more than five million shared project in its repository \footnote{http://scratch.mit.edu/statistics/}.
\paragraph{}Scratch is being used both in extracurricular activities \cite{maloney2008programming, kafai2010entering} or summer camps \cite{adams2010scratching, franklin2013assessment}, as in schools \cite{wilson2012evaluation}, high schools \cite{meerbaum2013learning} and even universities worldwide \cite{wolz2009starting, malan2007scratch}. Its capacity to bring computer programming to children and youth with the aim that in the future they would be willing to continue their studies in this discipline, and also its success to teach both basic and advanced programming concepts has been documented.
\paragraph{}However, it is also possible to find studies that have detected some bad programming habits in students learning to program in this environment \cite{meerbaum2011habits}. In this line, in our work as teachers we have detected other bad habits that, despite insisting our students to try to avoid them, we see how they are constantly repeated by pupils learning to program in Scratch. As these habits are contrary to the basic programming recommendations, our study tries to detect if they are also common in the projects shared on the Scratch repository and propose some ideas to try to avoid such situations.
\paragraph{}In section 2 a brief overview of Scratch is presented, works that study the learning demonstrated by students learning to program in this environment are reviewed, and studies that automate the evaluation of this learning are exposed. In section 3, our work methodology is described. Section 4 presents the obtained results. Finally, in section 5 the conclusions of our study are discussed and some ideas and suggestions for future work are provided.

\section{Background}

\paragraph{}Scratch is a visual programming environment developed by the Lifelong Kindergarten group at MIT Media Laboratory that allows to program interactive projects using a drag-and-drop approach with different pieces or \textit {Lego} style blocks \cite{resnick2009scratch}. Scratch was designed with the main goal of encouraging young people to program, and most of its users are novice learners with no prior programming experience \cite{maloney2010scratch}. The environment is designed to encourage users to manipulate and modify various media types, so that children and youth can create projects that they find interesting and attractive as animated stories, games or interactive presentations \cite{maloney2008programming}
\paragraph{}Scratch also offers a website where the projects programmed by the community are published, the code of each program can be studied, and it is possible to make suggestions and reuse the code of any project, so there are lots of valuable resources available for anyone willing to start to program. This approach also offers the opportunity to learn certain social skills related to programming, such as sharing and contributing to the community \cite{scaffidi2012skill}.
\paragraph{}As mentioned in the introduction, many papers showing that Scratch works as a tool for learning to program can be found \cite{maloney2008programming, franklin2013assessment, scaffidi2012skill, malan2007scratch}, as there are success stories in different educational levels, environments and disciplines showing that students learn concepts such as user interaction, conditional statements, communication and synchronization, logical expressions and data representation.
\paragraph{}Despite these good results, Meerbaum-Salant, Armoni and Ben-Ari \cite{meerbaum2011habits} prove that students learning to program in this environment show some bad habits that are contrary to recommended programming practices. In particular, they show excessive decomposition tendency and a development process absolutly from bottom to top starting with individual blocks of Scratch.
\paragraph{} Aiming to simplify the process of evaluating the Computational Thinking and help evaluators with an automation tool, Hairball was created \cite{boe2013hairball}. Hairball is a static analyzer of Scratch projects inspired by Lint \footnote{http:// www.unix.com/man-page/FreeBSD/1/lint} that tries to detect programming errors in Scratch projects, such as not initializing the value of a variable or the state of a character, sending messages that are not received by any program or including code that is never run.

\section{Methodology}

\paragraph{}En nuestro trabajo como docentes con estudiantes de secundaria, hemos detectado que muchos estudiantes presentan dos malos hábitos de manera recurrente, a pesar de nuestra constante insistencia para tratar de evitarlos, que tienen que ver con el nombrado de objetos y con la repetición de código.

\paragraph{}Por una parte, es muy habitual comprobar que los estudiantes no cambian el nombre de los personajes, que el entorno bautiza como \textit{SpriteX}, donde X es un número incremental, de manera que al crecer el número de personajes de un proyecto resulta complicado saber a qué objeto se refiere una determinada sentencia, contribuyendo así a una mala legibilidad del código, ralentizando la programación del proyecto y complicando su depuración ante posibles errores.
\paragraph{}Por otro lado, es también muy habitual observar a los estudiantes repetir código en un mismo proyecto, a veces incluso en los programas de un mismo personaje. De esta forma, además de complicar el mantenimiento y la actualización del código, no se trabajan los conceptos de abstracción y modularización, dos componentes fundamentales del desarrollo del pensamiento computacional \cite{wing2008computational}.
\paragraph{}Con el objetivo de detectar estos dos comportamientos en los proyectos Scratch, se han desarrollado dos plug-ins para el entorno Hairball \cite{boe2013hairball}, que a su vez hace uso de la biblioteca implementada en python \textit{kurt}\footnote{https://github.com/blob8108/kurt}, para automatizar la localización de este tipo de prácticas en programas desarrollados en este entorno.
\paragraph{}El plug-in \textit{convention.SpriteNaming}\footnote{https://github.com/jemole/hairball/blob/master/hairball/plugins/convention.py} analiza un proyecto Scratch para comprobar si los nombre de los distintos personajes que aparecen en el proyecto comienzan con el string \textit{Sprite}, lo que indica que el usuario no ha modificado este nombre y ha dejado el que el entorno asigna por defecto a cada nuevo personaje añadido al proyecto.
\paragraph{}El plug-in \textit{duplicate.DuplicateScripts}\footnote{https://github.com/jemole/hairball/blob/master/hairball/plugins/duplicate.py} analiza un proyecto Scratch para tratar de localizar programas completos repetidos dentro de un proyecto. Para ello recorre cada programa del proyecto y obtiene los \textit{tokens} de los bloques que lo forman, de manera que dos bloques que tan solo difieren en los valores que reciben son considerados iguales. Para ilustrar el funcionamiento de este plug-in, la imagen\ref{fig:CodeRepetition1} muestra dos scripts que contienen los mismos bloques con la única diferencia de los valores recibidos por parámetro. 
\begin{figure}
  \centering
    \includegraphics{img/CodeRepetition1.png}
  \caption{Two scripts repeating code}
  \label{fig:CodeRepetition1}
\end{figure}

\paragraph{}Estos scripts al ser analizados por el plug-in y traducidos a \textit{tokens} serían considerados iguales, ya que su código sería traducido al siguiente conjunto de bloques:
\begin{verbatim}
 'set rotation style %s'
 'repeat %s%s'
 'move %s steps'
 'next costume'
 'if on edge, bounce'
 'wait %s secs'
\end{verbatim}
\paragraph{} Así, la forma correcta de implementar este proyecto habría sido, tal como se muestra en la figura \ref{fig:CodeRepetition2}, definiendo un nuevo conjunto de bloques que recibe varios valores por parámetro y reutilizando el bloque creado en ambos programas, con las mejoras que esta solución implicaría en lo relativo al mantenimiento, actualización y depuración del código.

\begin{figure}
  \centering
    \includegraphics{img/CodeRepetition2.png}
  \caption{Defining own blocks to avoid code repetition}
  \label{fig:CodeRepetition2}
\end{figure}

\paragraph{}Para probar el funcionamiento de los plug-ins desarrollados y comprobar hasta qué punto estos comportamientos que habíamos detectado en nuestro trabajo se repiten también en los proyectos del repositorio de Scratch, procedimos a descargar 100 proyectos del mismo. Como la posibilidad de que los usuarios puedan definir sus propios bloques, con el uso del bloque \textit{def\_block}, solo se encuentra disponible desde el lanzamiento de Scratch 2.0, los proyectos descargados son todos posteriores a esta fecha.


\section{Findings}
\paragraph{}Los datos obtenidos tras analizar los 100 proyectos descargados confirman que las observaciones realizadas con nuestros estudiantes son también muy comunes en los proyectos de la comunidad. 
\paragraph{}En relación con la falta de costumbre de renombrar los personajes y no utilizar el nombre por defecto que asigna el entorno, el 79\% de los proyectos analizados contenían al menos un personaje sin renombrar del tipo \textit{SpriteX}.
\paragraph{}Como ejemplo para ilustrar los inconvenientes asociados a esta mala práctica se muestra en la imagen \ref{fig:SpriteNaming} un script que resulta difícilmente legible, ya que es necesario recorrer el área de personajes para localizar a qué personaje se refiere un bloque determinado, lo que dificulta la depuración del código ante eventuales problemas o mejoras e, incluso, ralentiza la propia programación por parte del estudiante, que debe memorizar o comprobar el número de objeto sobre el que quiere realizar una acción determinada.
\begin{figure}
  \centering
    \includegraphics{img/SpriteNaming.png}
  \caption{Sprites with default names}
  \label{fig:SpriteNaming}
\end{figure}
\paragraph{}En cuanto a la repetición de código, el 62\% de los proyectos analizados contenían al menos un programa repetido. De hecho, tan solo el 17\% de los proyectos analizados hacían uso del bloque \textit{def\_block} para definirse un bloque propio y poder reutilizarlo en otras partes del proyecto.

\section{Conclussions}

\paragraph{} En lo relativo al nombrado de objetos, es curioso comprobar que los usuarios de Scratch sí nombran correctamente, es decir, de forma significativa, las variables que utilizan en sus proyectos. En nuestra opinión esto se debe a que cuando se crea una variable es obligatorio asignarle un nombre, ya que el entorno no las nombra por defecto. Sin embargo, al crear un nuevo personaje se le asinga el nombre \textit{SpriteX} de manera automática. Quizás una modificación del entorno para que obligue al usuario a nombrar a los nuevos objetos eliminaría esta mala práctica.
\paragraph{} En relación a la repetición de código y el poco uso que los usuarios de Scratch hacen de la definición de bloques propios (el equivalente a la creación de métodos o procedimientos en otros lenguajes), quizás trabajos como el realizado por Seiter y Foreman \cite{seiter2013modeling} en el que se pone de manifiesto que la capacidad de abstracción y la modularización son conceptos y habilidadades que parecen desarrollarse a partir de cierta edad, puedan explicar los bajos índices de uso de esta funcionalidad, ya que una parte importante de la comunidad Scratch está formada por niños menores de 10 años\footnote{http://scratch.mit.edu/statistics/\#age}. Futuros trabajos deberían seguir investigando en esta línea para continuar avanzando en la definición de un marco de trabajo apropiado para distintas edades y para comprobar si los usuarios que aprenden en línea y aquellos que lo hacen en un entorno reglado demuestran índices distintos de uso de esta funcionalidad. 

\newpage
\bibliography{scratch}
\bibliographystyle{unsrt}
\end{document}
